\documentclass[]{friggeri-cv}
\usepackage{afterpage}
\usepackage{hyperref}
\usepackage{color}
\usepackage{xcolor}
\hypersetup{
    pdftitle={},
    pdfauthor={},
    pdfsubject={},
    pdfkeywords={},
    colorlinks=false,       % no lik border color
   allbordercolors=white    % white border color for all
}
\addbibresource{bibliography.bib}
\RequirePackage{xcolor}
\definecolor{pblue}{HTML}{0395DE}

\begin{document}
\header{Davi}{Mendes}
      {Engenheiro da Computação}

\fcolorbox{white}{gray}{\parbox{\dimexpr\textwidth-2\fboxsep-2\fboxrule}{
.....
}}

% In the aside, each new line forces a line break
\begin{aside}
  \section{Endereço}
    Rua Tereza Valverde Valério, 125
    Jd. Elite, Piracicaba, São Paulo
    Brasil
    ~
  \section{Fone}
    (19) 99189 2222
    ~
  \section{Mail}
    \href{mailto:ddioriomendes@gmail.com}{\textbf{ddioriomendes@}\\gmail.com}
    ~
  \section{Web \& Git}
    \href{https://github.com/ddmendes/}{github.com/ddmendes/}
    \href{https://br.linkedin.com/in/davim/}{linkedin.com/in/davim/}
    ~
  \section{Habilidades}
    \textbf{Java}\includegraphics[scale=0.40]{img/4stars.png}
    \textbf{Android}\includegraphics[scale=0.40]{img/4stars.png}
    \textbf{Git}\includegraphics[scale=0.40]{img/4stars.png}
    \textbf{Python}\includegraphics[scale=0.40]{img/3stars.png}
    \textbf{C/C++}\includegraphics[scale=0.40]{img/3stars.png}
    \textbf{\LaTeX}\includegraphics[scale=0.40]{img/3stars.png}
    \textbf{JavaScript}\includegraphics[scale=0.40]{img/3stars.png}
    \textbf{SQL}\includegraphics[scale=0.40]{img/3stars.png}
    \textbf{Matlab}\includegraphics[scale=0.40]{img/3stars.png}
    \textbf{FrontEnd}\includegraphics[scale=0.40]{img/2stars.png}
    \textbf{jQuery}\includegraphics[scale=0.40]{img/2stars.png}
    \textbf{Linux}\includegraphics[scale=0.40]{img/2stars.png}
    ~
  \section{Línguas}
    \textbf{Português}\includegraphics[scale=0.40]{img/5stars.png}
    \textbf{Inglês}\includegraphics[scale=0.40]{img/4stars.png}
\end{aside}

\section{Apresentação}
  Apaixonado por programação, tive meu primeiro contato através de uma revista de HTML,
  quando tinha 15 anos. Desde então fiz um curso de \emph{web designer} e um técnico em
  informática, até a graduação em engenharia da computação. Durante esta, eu trabalhei
  como monitor de disciplina e pesquisador. No quinto ano, estagiei no insituto de
  pesquisa da Samsung, trabalhando com Android, e na Circuitar, trabalhando com Arduino. Integrei ambas as tecnologias em meu trabalho de conclusão de curso, tendo uma nota
  9 de 10.

\section{Experiência}
\begin{entrylist}
  \entry
    {08/15 - Now}
    {Estagiário}
    {Circuitar Eletrônicos, Importação e Exportação, São Carlos (SP), Brasil}
    {Desenvolver e manter as bibliotecas de programação dos produtos.\\
     Manter tanto o \emph{back-end} (django) quanto o \emph{front-end} (bootstrap)
     da loja virtual da empresa.\\}
  \entry
    {01/15 - 07/15}
    {Auxiliar}
    {Samsung Instituto de Desenvolvimento para a Informática, Campinas (SP), Brasil}
    {Desenvolver protóripos para a inovação do time.\\}
    \entry
    {08/12 - 07/13}
    {Pesquisador}
    {Universidade de São Paulo, São Carlos (SP), Brasil}
    {Desenvolver um sistema para revisão de sentenças alinhadas automaticamente.\\
     Subprojeto de uma plataforma de estudo de tradução.\\}
\end{entrylist}

\section{Educação}
\begin{entrylist}
  \entry
    {2011 - 2015}
    {Bacharel em Engenharia da Computação}
    {Universidade de São Paulo, Brasil}
    {Área: Programação, Desenvolvimento de Sistemas, Sistemas Embarcados.\\
    \emph{Título do TCC: "SmartPendant, um periférico para smartphones Android".}\\}
\end{entrylist}

\end{document}